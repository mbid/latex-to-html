\documentclass[a4paper]{article}

\usepackage[utf8]{inputenc}
\usepackage[english]{babel}
\usepackage{amsfonts}
\usepackage{amsmath}
\usepackage{amsthm}
\usepackage{amssymb}
\usepackage{dsfont}
\usepackage{url}
\usepackage{hyperref}
\usepackage{tikz-cd}
\usepackage{mathtools}
\usepackage{color}
\usepackage{mathpartir}

\mathtoolsset{showonlyrefs}

\newcommand{\todo}[1]{\textcolor{red}{#1}}

\newtheorem{theorem}{Theorem}
\newtheorem{corollary}[theorem]{Corollary}
\newtheorem{lemma}[theorem]{Lemma}
\newtheorem{proposition}[theorem]{Proposition}
\theoremstyle{remark}
\newtheorem{remark}[theorem]{Remark}
\theoremstyle{definition}
\newtheorem{definition}[theorem]{Definition}

\begin{document}

\title{Eqlog: Datalog with functions and equality}

\author{Martin E. Bidlingmaier}

\date{}

\maketitle

\begin{abstract}
  We introduce eqlog, an extensions of datalog that supports function symbols and infers equalities.
  We show that eqlog encompasses precisely the finite essentially algebraic theories.
  An eqlog engine is then a tool to perform computations in (fragments of) free models of essentially algebraic theories.

  In the second part, we show how the semi-naive evaluation strategy of datalog can be combined with the standard congruence closure algorithm into an efficient algorithm for eqlog evaluation.
  We discuss several optimizations, and show how an eqlog can replace both standard datalog and e-graphs for their respective applications without algorithmic slowdown.

  The paper is accompanied by a tool that translates eqlog theories to specialized rust code.
\end{abstract}

\section{Essentially algebraic theories and free models}

In this section we recall the notion of \emph{(finite) essentially algebraic theory} and give a brief proof sketch of the existence of free models.
Our proof proceeds by associating to each axiom $T$ of an essentially algebraic theory an epimorphic morphism $[T]$ in the category of structures for the signature of the theory.
We show that the models of the theory are precisely the orthogonal structures.
From this it follows via the orthogonal-reflection construction that a free model exists over every structure.

We then discuss relational signatures and theories.
Relational signatures do not have function symbols.
Every algebraic signature induces a relational signature by associating to every function symbol a relation symbol representing the graph.
The algebraic structures can then again be obtained as an orthogonality class in the category of relational structures.

We then introduce relational theories.
We again associate morphism $[T]$ of relational structures to every relational axiom $T$.
In contrast to the algebraic case, $[T]$ need not always be an epimorphism, so the orthogonal-reflection construction does not directly apply.
However, we show how algebraic theories can be encoded as relational theories, for which an adaptation of the orthogonal-reflection reaches a fixed point.
This construction is the basis of the eqlog algorithm.

\subsection{The orthogonal-reflection construction}
\label{subsec:orthogonal-reflection}

Fix a cocomplete category $\mathcal{C}$ for the remainder of Section \ref{subsec:orthogonal-reflection}.

\begin{definition}
  Let $f : A \rightarrow B$ be a morphism let $X$ be an object.
  We write $f \pitchfork X$ and say that $X$ is \emph{injective} to $f$ if for all maps $a : A \rightarrow X$, there exists a map $b : B \rightarrow X$ such that
  \begin{equation}
    \begin{tikzcd}
      A \arrow[r, "a"] \arrow[d, "f"'] & X \\
      B \arrow[ur, dashed, "\exists b"']
    \end{tikzcd}
  \end{equation}
  commutes.
  If furthermore $b$ is unique for all $a$, then we write $f \perp X$ and say that $X$ is \emph{orthogonal} to $f$.

  If $M$ is a class of morphisms, then we write $M \pitchfork X$ if $f \pitchfork X$ for all $f \in M$, and $M \perp X$ if $f \perp X$ for all $f \in M$.
  The full subcategories given by the injective and orthogonal objects, respectively, are denoted by $M^\pitchfork$ and $M^\perp$.
  % This notation is also used in Adamek-Rosicky.
\end{definition}

\begin{definition}
  \label{def:flub}
  A class $M$ of morphisms is called \emph{flub} \todo{placeholder} if $M^\pitchfork = M^\perp$.
\end{definition}

One of the main sources of flub sets is the following Lemma:
\begin{lemma}
  Let $M$ be a class of epimorphisms.
  The $M$ is flub.
  \qed
\end{lemma}

The other one is the following proposition.
It lets us reduce questions about orthogonality classes to flub injectivity classes.
\begin{proposition}
	Let $M$ be a class of morphisms.
	Then there exists a superclass $N \supseteq M$ such that $N$ is flub and $N^\pitchfork = M^\perp$.
  If $M$ is a set, then $N$ can be chosen as set.
\end{proposition}
\begin{proof}
  Let $f : A \rightarrow B$ be a morphism in $M$.
  Then for each object $X$, the data of a single map $a : A \rightarrow X$ and two maps $b_1, b_2 : B \rightarrow X$ such that
  \begin{equation}
    \begin{tikzcd}
      A \arrow[r, "a"] \arrow[d, "f"'] & X \\
      B \arrow[ur, "b_i"'] &
    \end{tikzcd}
  \end{equation}
  commutes for $i \in \{1, 2\}$ is in bijective correspondence to a map $\langle b_1, b_2 \rangle : B \amalg_A B \rightarrow X$.
  Let
  \begin{equation}
    f_\mathrm{eq} : B \amalg_A B \rightarrow B.
  \end{equation}
  be the canonical map that collapses the two copies of $B$ into one.
  Then $b_1 = b_2$ if and only if there exists a map $b$ such that
  \begin{equation}
    \begin{tikzcd}
      B \amalg_A B \arrow[r, "{\langle b_1, b_2\rangle}"] \arrow[d, "f_\mathrm{eq}"'] & X \\
      B \arrow[ur, "b"'] &
    \end{tikzcd}
  \end{equation}
  commutes.
  The map $f_\mathrm{eq}$ is an epimorphism.
  Thus if $b$ exists, then it exists uniquely, and $b = b_1 = b_2$.
  It follows that $X$ is orthogonal to $f$ if and only if $f$ is injective to both $f$ and $f_\mathrm{eq}$.
  The desired set $N$ can thus be defined by $N = M \cup \{ f_\mathrm{eq} \mid f \in M \}$.
\end{proof}

\begin{definition}
  Let $M$ be a class of morphisms.
  A \emph{relative $M$-complex} is (isomorphic to) a transfinite composition of pushouts of morphisms in $M$.
  The class of relative $M$-complexes is denoted by $\mathrm{Cell}(M)$.
\end{definition}

\begin{proposition}
  Let $M$ be a class of morphisms.
  Relative $M$-cell complexes satisfy the following closure properties:
  \begin{enumerate}
    \item
      $\mathrm{Cell}(M)$ contains all isomorphisms.
    \item
      $M \subseteq \mathrm{Cell}(M)$.
    \item
      \label{itm:cell-idempotent}
      $\mathrm{Cell}(\mathrm{Cell}(M)) = \mathrm{Cell}(M)$.
    \item
      \label{itm:cell-coproducts}
      $\mathrm{Cell}(M)$ is closed under coproducts.
    \item
      \label{itm:cell-injective}
      $\mathrm{Cell}(M)^\pitchfork = M^\pitchfork$.
    \item
      $\mathrm{Cell}(M)^\perp = M^\perp$.
      \qed
  \end{enumerate}
\end{proposition}
\begin{proof}
  We show \ref{itm:cell-idempotent}, \ref{itm:cell-coproducts} and \ref{itm:cell-injective}, and leave proofs of the other clauses to the interested reader.

  \ref{itm:cell-idempotent}.
  It suffices to show that $\mathrm{Cell}(M)$ is closed under pushouts and transfinite compositions.
  Let $f : X \rightarrow Y$ be in $\mathrm{Cell}(M)$, and let $k : X \rightarrow X'$ be an arbitrary map.
  By definition of $\mathrm{Cell}(M)$, we find a transfinite sequence of morphisms $m_0, m_1, \dots \in M$ and pushout squares square as in the top half of the following diagram such that $f$ is the transfinite composition of the $f_i$:
  \begin{equation}
    \begin{tikzcd}
      A_0 \arrow[r, "m_0"] \arrow[d] & B_0 \arrow[d] & A_1 \arrow[d] \arrow[r, "m_1"] & B_1 \arrow[d] & \dots \\
      X \arrow[d, "k"'] \arrow[r, "f_0"] & X_1 \arrow[d] \arrow[r, "="] & X_1 \arrow[d] \arrow[r, "f_1"] & X_2 \arrow[d] \arrow[r] & \dots \arrow[r] & Y \arrow[d, "k'"] \\
      X' \arrow[r, "f'_0"] & X'_1 \arrow[r, "="] & X'_1 \arrow[r, "f'_1"] & X'_2 \arrow[r] & \dots \arrow[r] & Y'
    \end{tikzcd}
  \end{equation}
  Define $f'_0, f'_1, \dots$ for successor ordinals such that the squares in the bottom half are pushouts, and for limit ordinals as transfinite composition of the previous $f'_i$.
  Because pushouts commute with transfinite composition, it follows that each horizontal rectangle in the bottom half of the diagram is a pushout.
  In particular, the map $X' \rightarrow Y'$ is a pushout of $f$ along $k$.
  By the pasting lemma for pushouts, the vertical rectangles with edges $A_i, B_i, X'_i, X'_{i + 1}$ are pushouts.
  Thus each map $f'_i$ is a pushout of a map in $M$, and $f'$ is the transfinite composition of the $f'_i$.

  For closure of $\mathrm{Cell}(M)$, it suffices to show that a transfinite compomsition of transfinite compositions can be reduced to a single transfinite composition.
  Thus let $(f_a)_{a \in \alpha}$ be a transfinite sequence of maps indexed by an ordinal $\alpha$ such that for all $a \in \alpha$, the map $f_a$ is the transfinite composition of a sequence $g_{a, b}$ for some ordinal $\beta_a$.
  Set
  \begin{equation}
    \gamma = \beta_0 + \beta_1 + \dots \cong \{ (a, b) \mid a \in \alpha, b \in \beta_a \}
  \end{equation}
  where the order is given by
  \begin{equation}
    (a_0, b_0) \leq (a_1, b_1) \iff \begin{cases}
      a_0 \leq a_1 & \text{if } a_0 \neq a_1 \\
      b_0 \leq b_1 & \text{otherwise}
    \end{cases}
  \end{equation}
  Then $\gamma$ is well-ordered, and the transfinite composition of the $(f_a)_{a \in \alpha}$ can be computed as transfinite composition of $(g_{a, b})_{(a, b) \in \gamma}$.

  \ref{itm:cell-coproducts}.
  We consider first binary coproducts of maps $f_0 : A_0 \rightarrow B_0$ and $f_1 : A_1 \rightarrow B_1$ in $\mathrm{Cell}(M)$.
  Consider the following diagram:
  \begin{equation}
    \begin{tikzcd}
      A_0 \arrow[d] \arrow[r, "f_0"] & B_0 \arrow[d] & A_1 \arrow[d] \arrow[r, "f_1"] & B_1 \arrow[d] \\
      A_0 \amalg A_1 \arrow[r, "f_0 \amalg \mathrm{id}"] & B_0 \amalg A_1 \arrow[r, "="] & B_0 \amalg A_1 \arrow[r, "\mathrm{id} \amalg f_1"] & B_0 \amalg B_1
    \end{tikzcd}
  \end{equation}
  Both squares are pushout squares, and the composition $A_0 \amalg A_1 \rightarrow B_0 \amalg B_1$ equals $f_0 \amalg f_1$.
  By \ref{itm:cell-idempotent}, it follows that $f_0 \amalg f_1 \in \mathrm{Cell}(M)$.
  This can be generalized to arbitrary coproducts over potentially infinite sets $I \subseteq \mathrm{Cell}(M)$ by first choosing an ordinal $\alpha \cong I$, and then computing $\coprod I$ as a transfinite composition of a sequence indexed by $\alpha$.

  \ref{itm:cell-injective}
  The inclusion $M \subseteq \mathrm{Cell}(M)$ immediately implies that every object that is injective to $\mathrm{Cell}(M)$ is also injective to $M$.
  Conversely, let $X$ be an object and let $N = \{f \in \operatorname{Mor} \mathcal{C} \mid f \pitchfork X \}$.
  We show that $N$ is closed under pushouts and transfinite composition.

  For pushouts, consider the diagram
  \begin{equation}
    \begin{tikzcd} \\
      A \arrow[r] \arrow[d, "m"'] \arrow[dr, very near end, phantom, "\ulcorner" description] & A' \arrow[d, "m'"] \arrow[r] & X \\
      B \arrow[r] & B'
    \end{tikzcd}
  \end{equation}
  where $m \in M$ and $a, a'$ are arbitrary maps.
  Since $m \pitchfork X$, there exists a lift $b : B \rightarrow X$, and then because of the universal property of pushouts the lift extends uniquely to a lift $b' : B' \rightarrow X$ that commutes with $b$.
  Thus $m' \pitchfork X$.
\end{proof}

\begin{lemma}
  \label{lem:factoring-via-cell-complexes}
  Let $M$ be a class of morphisms.
  Let $f : X \rightarrow Y$ be a relative $M$-cell complex, and let $g : X \rightarrow Z$ be a map with $Z \in M^\pitchfork$.
  Then there is a map $h : Y \rightarrow Z$ such that $hf = g$.
  If furthermore $M$ is flub, then $h$ is unique.
\end{lemma}
\begin{proof}
\end{proof}

\begin{proposition}
  Let $M$ be flub class of morphisms.
  Let $f : X \rightarrow Y$ be a relative $M$-cell complex such that $Y \in M^\perp$.
  Then $f$ is a reflection of $X$ into $M^\pitchfork$.
\end{proposition}
\begin{proof}
  By Lemma \ref{lem:factoring-via-cell-complexes}.
\end{proof}

\begin{definition}
  \label{def:finitely-presentable}
  An object $X$ is \emph{finitely presentable} if the hom-functor $\mathrm{Hom}(X, -) : \mathcal{C} \rightarrow \mathrm{Set}$ preserves filtered colimits.
\end{definition}
\todo{Make this explicit maybe.}

\begin{proposition}[Orthogonal-reflection construction]
  \label{prop:orth-refl-constr-prop}
  Let $M$ be a flub class of morphisms with finitely presentable domains and codomains.
  Let
  \begin{equation}
    X_0 \xrightarrow{x_0} X_1 \xrightarrow{x_1} \dots
  \end{equation}
  be a countable chain of morphisms in $M$ such that the following holds:
  \begin{enumerate}
    \item
      $x_n$ is a relative $M$-cell complex for all $n$.
    \item
      \label{itm:m-factorizations-exist}
      For all $f : A \rightarrow B$ in $M$, $n \geq 0$ and maps $a : A \rightarrow X_n$, there exists a map a map $b : B \rightarrow X_m$ for some $m \geq n$ such that
      \begin{equation}
        \begin{tikzcd}
          A \arrow[rrr, "f"] \arrow[d, "a"'] & & & B \arrow[d, "b"] \\
          X_n \arrow[r, "x_n"] & X_{n + 1} \arrow[r, "x_{n + 1}"] & \dots \arrow[r, "x_{m - 1}"] & X_m
        \end{tikzcd}
      \end{equation}
      commutes.
  \end{enumerate}
  Then the transfinite composition $X_0 \rightarrow \operatorname{colim}_{n \geq 0} X_n$ is a reflection into $M^\pitchfork$.
\end{proposition}
\begin{proof}
  The transfinite composition of the relative $M$-cell complexes $x_n$ is a relative $M$-cell complexes by Lemma \todo{REF}.
  Thus by Lemma \todo{REF}, it suffices to show that $\bar X = \operatorname{colim}_{n \geq 0}$ is in $M$-injective.

  Let $f : A \rightarrow B$ be in $M$ and let $a : A \rightarrow \bar X$.
  Because $A$ is finitely presentable, there exists $n$ and $a_n : A \rightarrow X_n$ such that $a$ factors as $A \rightarrow X_n \rightarrow \bar X$.
  By assumption \ref{itm:m-factorizations-exist}, there exist $m$ and $b_m : B \rightarrow X_m$ that commutes with $f$, $a_n$ and $x_{m - 1} \circ \dots \circ x_n$.
  Thus if we define $b$ as composition $B \rightarrow X_m \rightarrow \bar X$, then $a = b \circ f$.
\end{proof}

\begin{proposition}
  \label{prop:orth-refl-constr-exist}
  Let $M$ be flub set (not a proper class!) of morphism and let $X$ be an object.
  Then there exists a chain
  \begin{equation}
    X_0 \xrightarrow{x_0} X_1 \xrightarrow{x_1} \dots
  \end{equation}
  satisfying the conditions of Proposition \ref{prop:orth-refl-constr-exist} such that $X = X_0$.
  In particular, $X$ has a reflection in $M^\pitchfork$, which is a reflective subcategory of $\mathcal{C}$.
\end{proposition}
\begin{proof}
  It suffices to construct a relative $M$-cell complexes $X \rightarrow Y$ such that for every $f : A \rightarrow B$ in $M$ and $a : A \rightarrow X$, there exists a commuting diagram
  \begin{equation}
    \begin{tikzcd}
      A \arrow[r, "f"] \arrow[d, "a"'] & B \arrow[d, "b"] \\
      X \arrow[r] & Y.
    \end{tikzcd}
  \end{equation}
  The required chain can then be obtained by induction.

  Let $I$ be the set of pairs $(f, a)$, where $f : A \rightarrow B$ is a morphism in $M$ and $a : A \rightarrow X$.
  Note that $I$ is a set because $M$ is a set and $\mathrm{Hom}(A, X)$ is a set for all $A$.
  We then define $X \rightarrow Y$ by the following pushout diagram:
  \begin{equation}
    \begin{tikzcd}
      \coprod_{(f, a) \in I} \operatorname{dom} f \arrow[r] \arrow[d] \arrow[dr, phantom, very near end, "\ulcorner"] & \coprod_{(f, a) \in I} \operatorname{cod} f \arrow[d] \\
      X \arrow[r] & Y
    \end{tikzcd}
  \end{equation}
  Here the top map is given by the coproduct of the first projection $f$ for each $(f, a) \in I$, and the top map is given by the second component $a$ for each $(f, a) \in I$.
\end{proof}

\subsection{Relational and algebraic structures}

\begin{definition}
  A \emph{relational signature} is given by the following data:
  \begin{itemize}
    \item
      A set $S$ of \emph{sort symbols}.
    \item
      A set $R$ of \emph{relation symbols}.
    \item
      A map that assigns to each relation symbol $r \in R$ an \emph{arity}
      \begin{equation}
        r : s_1 \times \dots \times s_n
      \end{equation}
      of sort symbols $s_1, \dots, s_n \in S$ for $n \geq 0$.
  \end{itemize}
  
  An \emph{algebraic signature} is a relational signature $(S, R)$ equipped with a partition $R = P \sqcup F$ of the set of relation symbol into disjoint sets $P$ of \emph{predicate symbols} and $F$ of \emph{function symbols} such that the arity of every function symbol is non-empty.
  If $f \in F$ is a function symbol, then we write $f : s_1 \times \dots \times \dots s_n \rightarrow s$ if the arity of $f$ as a relation symbol is $f : s_1 \times \dots \times s_n \times s$.
\end{definition}

\begin{definition}
  Let $\mathcal{S} = (S, R)$ be a relational signature.
  A \emph{relational structure} for $\mathcal{S}$ consists of the following data:
  \begin{itemize}
    \item
      For each sort symbol $s \in S$, a \emph{carrier} set $X_s$.
    \item
      For each relation symbol $r \in R$ with arity $r : s_1 \times \dots \times s_n$, a relation $r_X \subseteq X_{s_1} \times \dots \times X_{s_n}$.
  \end{itemize}

  A \emph{morphism of relational structures} $f : X \rightarrow Y$  consists of a function $f_s : X_s \rightarrow Y_s$ that are compatible with the relations $r_X$ and $r_Y$ for all $r$.
  That is, we require that if $(x_1, \dots, x_n) \in r_X$ for some relation symbol $r : s_1 \times \dots \times s_n$, then $(f_{s_1}(x_1), \dots, f_{s_n}(x_n)) \in r_Y$.
  The \emph{category of relational structures} is denoted by $\mathrm{Rel}_\mathcal{S}$.

  Let $\mathcal{S} = (S, P \sqcup F)$ be an algebraic signature.
  An \emph{algebraic structure} for $\mathcal{S}$ is a relational structure $X$ for $\mathcal{S}$ such that $f_X$ is the graph of a partial function for all $f \in F$.
  Thus if $(x_1, \dots, x_n, y) \in f_X$ and $(x_1, \dots, x_n, z) \in f_X$, then $y = z$.
  We use $f_X(x_1, \dots, x_n)$ to denote the unique element $y$
 such that $(x_1, \dots, x_n, y) \in f_X$, and we write $f_X(x_1, \dots, x_n) \downarrow$ to say that such an element $y$ exists.

  A \emph{morphism of algebraic structures} is a morphism of underlying relational structures.
  The \emph{category of algebraic structures} is denoted by $\mathrm{Alg}_\mathcal{S}$.
\end{definition}

When no confusion can arise, we suppress sort annotations.
Thus if $X$ is a relational structure, then we write $x \in X$ to mean that $x \in X_s$ for some $s \in S$.
Similarly, if $f : X \rightarrow Y$ is a morphism of relational structures and $x \in X_s$, then we often denote the image of $x$ under $f$ by $f(x)$ instead of $f_s(x)$.

If the signature $\mathcal{S}$ is clear from context, we abbreviate $\mathrm{Rel}_\mathcal{S}$ as $\mathrm{Rel}$ and $\mathrm{Alg}_\mathcal{S}$ as $\mathrm{Alg}$.

\begin{proposition}
  \label{prop:rel-cocomplete}
  Let $\mathcal{S} = (S, R)$ be a relational signature.
  All colimits of relational structures exist, and commute with colimits on carrier sets.
\end{proposition}
\begin{proof}
  Let $D : I \rightarrow \mathrm{Rel}_\mathcal{S}$ be a diagram of relational structures.
  We define the carrier sets of our candidate colimit structure $X$ by the colimit of carrier sets.
  That is,
  \begin{equation}
    X_s = \operatorname*{colim}_{i \in I} D(i)_s
  \end{equation}
  for all $s \in S$.
  We obtain evident maps $(p_i)_s : D(i)_s \rightarrow X_s$ for all objects $i$ in $I$ and $s \in S$.
  Let $r : s_1 \times \dots \times s_n$ be a relation symbol.
  Then we define $r_X$ as union over the images of the $r_{D(i)}$.
  Thus,
  \begin{equation}
    r_X = \bigcup_{i \in I} p_i(r_{D(i)})
  \end{equation}
  where we abbreviated $p_i(r_{D(i)}) = ((p_i)_{s_1} \times \dots \times (p_i)_{s_n})(r_{D(i)})$.
\end{proof}

\begin{proposition}
  Let $\mathcal{S} = (S, R)$ be a relational signature.
  Then a relational structure $X$ is finitely presentable if and only if
  \begin{equation}
    \sum_{s \in S} |X_s| < \infty,
  \end{equation}
  that is, if and only if $X_s$ is empty for almost all $s$ and finite for all other $s$.
\end{proposition}
\begin{proof}
  The finitely presentable objects in $\mathrm{Set}$ are the finite sets.
  An object $(X_i)_{i \in I}$ in a product category $\prod_{i \in I} \mathcal{C}$ is finitely presntable if and only if $X_i$ is finitely presentable in $\mathcal{C}_i$ for all $i$.
  This proves the claim where $R = \emptyset$, because $\mathrm{Rel}_{(S, \emptyset)} \simeq \prod_{s \in S} \mathrm{Set}$.
  Finally, the general case follows from the fact that the forgetful functor $\mathrm{Rel}_\mathcal{S} \rightarrow \mathrm{Rel}_{(S, \emptyset)}$ is full and faithful and preserves colimits (Proposition \ref{prop:rel-cocomplete}).
  \todo{
    No, this functor is not full.
    The proposition is actually not true as stated:
    We must also have that almost all relations are empty.
    Otherwise $\operatorname{colim}_i \mathrm{Hom}(X, D(i)) \rightarrow \mathrm{Hom}(X, \operatorname{colim} D)$ need not be surjective (it is injective though).
  }
\end{proof}
\todo{
  Should call finitely presentable relational structures simply finite.
  Note that finitely presentable algebraic structures need not be finite as relational structures.
}

\begin{proposition}
  \label{prop:surjective-orthogonal-reflection}
  Let $\mathcal{S} = (S, R)$ be a finite relational signature.
  Let $M$ be a finite set of epimorphisms of finitely presentable relational structures.
  Let
  \begin{equation}
    X_0 \xrightarrow{x_0} X_1 \xrightarrow{x_1} \dots
  \end{equation}
  be any sequence of maps of relational structures satisfying the conditions of Proposition \ref{prop:orth-refl-constr-prop} such that furthermore $X_0$ is finitely presentable.
  Then the sequence is eventually stationary.
  That is, there exists $m$ sucht that for all $n \geq m$, the transition map $x_n$ is an isomorphism.
\end{proposition}
\begin{proof}
  Since all maps in $M$ are surjective and colimits of relational structures commute with colimits on carrier sets, it follows that all maps in $\mathrm{Cell}(M)$ are surjective.
  Thus the cardinality of the carriers of the $X_n$ decreases monotonically with $n$.
  Eventually, the cardinality must thus become stable, say after $n_0 \in \mathbb{N}$.
  Without loss of generality, we may assume that $x_n$ is the identity map on carriers for $n \geq n_0$.
  Let $r \in R$.
  Then for all $n \geq n_0$, we have that $r_{X_n} \subseteq r_{X_{n + 1}} \subseteq (X_{n_0}){s_1} \times \dots \times (X_{n_0})_{s_n}$, and the latter is a finite set.
  Thus, we eventually have $r_{X_n} = r_{X_{n + 1}}$.
  Because $R$ is finite, this holds simultaneously for all $r \in R$ for large enough $n_1 \geq n_0$.
  For $n \geq n_1$ we thus have $X_n = X_{n + 1}$.
\end{proof}

\begin{remark}
  Proposition \ref{prop:surjective-orthogonal-reflection} can likely be generalized to locally finitely presentable categories, as such categories are always co-wellpowered:
  For every object $X$, there exists up to isomorphism only a \emph{set} of epimorphism $X \twoheadrightarrow Y$.
  One now has to identify objects for which this set is finite.
\end{remark}

\begin{proposition}
  Let $\mathcal{S} = (S, P \sqcup F)$ be an algebraic signature.
  Then there exists a set $M$ of epimorphisms of relational structures such that the relational structures are algebraic if and only if they are injective to $M$.
\end{proposition}
\begin{proof}
  \todo{
    Maybe postpone the proof until after we have introduced relational sequents?
  }
  Let $f : s_1 \times \dots \times s_n \times s$ be a function symbol.
  We construct an epimorphism $m_f : A_f \rightarrow B_f$ of relational structures such that relational structures $X$ are orthogonal to $m_f$ if and only if $f_X$ is the graph of a partial function.
  Then $M = \{ m_f \mid f \in F \}$ is a set of morphisms with the required property.

  $A_f$ is the relational structure with carriers given by distinct symbols $c_1, \dots, c_n, d, e$, where $c_i$ has sort $s_i$ and both $d$ and $e$ have sort $s$.
  $r_{A_f}$ is empty for all $r \neq f$, and
  \begin{equation}
    f_{A_f} = \{ (c_1, \dots, c_n, d), (c_1, \dots, c_n, e) \}.
  \end{equation}
  Let $B_f$ be the quotient of $A_f$ given by identifying $d$ and $e$, and let $m_f : A_f \rightarrow B_f$ be the evident quotient map.

  The data of a map $a : A_f \rightarrow X$ to a relational structure $X$ is equivalent to elements $x_1, \dots, x_n, y, z$ in the carriers of $X$ such that $r_X(x_1, \dots, x_n, y)$ and $r_X(x_1, \dots, x_n, z)$.
  A factorization $b : B_f \rightarrow X$ exists (necessarily uniquely) if and only if $y = z$.
  Thus if all such factorizations exist, then $f_X$ is the graph of a partial function.
\end{proof}

\begin{corollary}
  Let $\mathcal{S} = (S, P \sqcup F)$ be an algebraic signature.
  The category of algebraic structures is a reflective subcategory of the category of relational structures.
  \qed
\end{corollary}

We denote the free algebraic structure functor by $\mathrm{FAlg} : \mathrm{Rel} \rightarrow \mathrm{Alg}$.

\begin{corollary}
  Let $\mathcal{S} = (S, P \sqcup F)$ be an algebraic signature.
  The category of algebraic structures is cocomplete.
\end{corollary}
\begin{proof}
  As in every reflective subcategory, colimits of algebraic structures can be computed by first computing the colimit of relational structures and then applying the free-algebra functor.
\end{proof}

\subsection{Relational theories and essentially algebraic theories}

Fix a relational signature $(S, R)$.
We assume a countable supply of variable symbols for each sort $s \in S$.
\begin{definition}
  \label{def:relational-syntax}
  A \emph{relational atom} is a statement of one of the following forms:
  \begin{enumerate}
    \item
      \label{itm:relation-atom}
      $r(x_1, \dots, x_n)$, where $r : s_1 \times \dots \times s_n$ is a relation symbol and the $v_i$ are variables of sort $s_i$ for all $i \in \{1, \dots, n\}$.
    \item
      $x \downarrow$ where $x$ is a variable.
    \item
      $x \equiv y$, where $x$ and $y$ are variables of the same sort.
  \end{enumerate}
  A \emph{relational formula} is a finite conjunction of flat atoms.
  A \emph{relational sequent} is an implication of flat formulas.
\end{definition}

\begin{definition}
  Let $\mathcal{S} = (S, P \sqcup F)$ be an algebraic signature.
  The set of \emph{terms} and a sort assigned to each term is given by the following recursive definition:
  \begin{enumerate}
    \item
      If $x$ is a variable of sort $s$, then $x$ is a term of sort $s$.
    \item
      If $f : s_1 \times \dots \times s_n \rightarrow s$ is a function symbol and $t_1, \dots, t_n$ are terms such that $t_i$ has sort $s_i$ for all $i = 1, \dots, n$, then $f(t_1, \dots, t_n)$ is a term of sort $s$.
  \end{enumerate}
  \emph{Algebraic atoms, formulas and sequents} are defined as in definition \ref{def:relational-syntax}, with two changes:
  \begin{enumerate}
    \item
      In each type of atom, also composite terms of the same sort are allowed in place of only variables.
    \item
      An algebraic atom $r(t_1, \dots, t_n)$ is valid only if $r = p$ is a predicate symbol (but not if $r$ is a function symbol).
  \end{enumerate}
  \todo{Maybe easier to just state it explicitly.}
\end{definition}

\begin{definition}
  \label{def:relational-semantics}
  Let $X$ be a relational structure.
  An \emph{interpretation of a set of variables $V$} in $X$ is a map $\mathcal{I}$ that assigns to each variable $v \in V$ of sort $s$ an element $\mathcal{I}(v) \in X_s$.
  An \emph{interpretation of a relational atom $\phi$} in $X$ is an interpretation $\mathcal{I}$ of the variables occuring in $\phi$ such that one of the following conditions holds:
  \begin{enumerate}
    \item
      $\phi = r(x_1, \dots, x_n)$ for some relation symbol $r$ and $(\mathcal{I}(x_1), \dots, \mathcal{I}(x_1)) \in r_X$.
    \item
      $\phi = x \downarrow$ for some variable $x$.
    \item
      $\phi = x \equiv y$ for some variables $x$ and $y$, and $\mathcal{I}(x) = \mathcal{I}(y)$.
  \end{enumerate}
  An \emph{interpretation of a relational formula $\mathcal{F} = \phi_1 \land \dots \land \phi_n$} in $X$ is an interpretation of the variables occuring in $\mathcal{F}$ that restricts to an interpretation of $\phi_i$ for each $i \in \{1, \dots, n\}$.

  A relational structure $X$ \emph{satisfies} a relational sequent $\mathcal{F} \Rightarrow \mathcal{G}$ if each interpretation of $\mathcal{F}$ in $X$ can be extended to an interpretation of $\mathcal{F} \land \mathcal{G}$ in $X$.
\end{definition}

\begin{definition}
  Let $\mathcal{S} = (S, P \sqcup F)$ be an algebraic signature and let $X$ be an algebraic structure.

  An \emph{interpretation of a term $t$} in $X$ is an interpretation $\mathcal{I}$ of the variables occuring in $t$ such that the following recursive extension of $\mathcal{I}$ is well-defined on $t$:
  \begin{equation}
    \mathcal{I}(f(t_1, \dots, t_n)) = f_X(\mathcal{I}(t_1), \dots, \mathcal{I}(t_n)).
  \end{equation}
  Note that the right-hand side might not be defined; in this case also the left-hand side is undefined.

  An \emph{interpretation of an algebraic atom} in $X$ is defined analogously to the interpretation of a relational atom, but with the additional condition that the interpretation is defined on all (possibly composite) terms occuring in the atom.

  An \emph{interpretation of an algebraic formula $\mathcal{F} = \phi_1 \land \dots \land \phi_n$} is an interpretation of the variables occuring in $\mathcal{F}$ that restricts to an interpretation of $\phi_i$ for each $i \in \{1, \dots, n\}$.

  An algebraic structure $X$ \emph{satisfies} an algebraic sequent $\mathcal{F} \Rightarrow \mathcal{G}$ if each interpretation of $\mathcal{F}$ in $X$ can be extended to an interpretation of $\mathcal{F} \land \mathcal{G}$ in $X$.
\end{definition}

\begin{lemma}
  \label{lem:functionality-axioms}
  Let $\mathcal{S} = (S, P \sqcup F)$ be an algebraic signature and let $X$ be a relational structure.
  Then $X$ is an algebraic structure if and only if it satisfies the relational sequent
  \begin{equation}
    \label{eq:functionality-axiom}
    f(v_1, \dots, v_n, u_0) \land f(v_1, \dots, v_n, u_1) \Rightarrow u_0 = v_1
  \end{equation}
  for each function symbol $f : s_1 \times \dots \times s_n \rightarrow s$.
\end{lemma}
\begin{proof}
  The relational structure $X$ satisfies the sequent \eqref{eq:functionality-axiom} if and only if $f_X$ is right-unique, i.e. the graph of a partial function.
\end{proof}

\begin{definition}
  Let $\mathcal{S} = (S, P \sqcup F)$ be an algebraic signature.

  Let $t$ be a term.
  The \emph{flattening} of $t$ consists of a relational formula $\mathrm{Flat}(t)$ and a result variable $v_\mathrm{Flat}(t)$.
  Flattening is defined recursively as follows:
  \begin{enumerate}
    \item
      If $t = v$ is a variable, then $\mathrm{Flat}(t) = \top$ is the empty conjunction and $v_\mathrm{Flat}(t) = v$.
    \item
      \label{itm:flattening-composite}
      If $t = f(t_1, \dots, t_n)$, then
      \begin{equation}
        \mathrm{Flat}(t) = \mathrm{Flat}(t_1) \land \dots \land \mathrm{Flat}(t_n) \land f(v_\mathrm{Flat}(t_1), \dots, v_\mathrm{Flat}(t_n), u)
      \end{equation}
      where $u$ is a fresh variable, and $v_\mathrm{Flat}(t) = u$.
  \end{enumerate}

  Let $\phi$ be an algebraic atom.
  The flattening $\mathrm{Flat}(\phi)$ is a relational formula which is defined depending on the type of $\phi$ as follows:
  \begin{enumerate}
    \item
      If $\phi = p(t_1, \dots, t_n)$ for some predicate $p$, then
      \begin{equation}
        \mathrm{Flat}(\phi) = \mathrm{Flat}(t_1) \land \dots \land \mathrm{Flat}(t_n) \land p(v_\mathrm{Flat}(t_1), \dots, v_\mathrm{Flat}(t_n)).
      \end{equation}
    \item
      If $\phi = t \downarrow$ for some term $t$, then
      \begin{equation}
        \mathrm{Flat}(\phi) = \mathrm{Flat}(t) \land v_\mathrm{Flat}(t) \downarrow.
      \end{equation}
    \item
      If $\phi$ is of the form $t_1 \equiv t_2$ for terms $t_1, t_2$, then
      \begin{equation}
        \mathrm{Flat}(\phi) = \mathrm{Flat}(t_1) \land \mathrm{Flat}(t_2) \land v_\mathrm{Flat}(t_1) \equiv v_\mathrm{Flat}(t_2).
      \end{equation}
  \end{enumerate}
  The flattening of an algebraic formula is the conjunction of the flattenings of each atom making up the formula.
  The flattening of an algebraic sequent is given by flattening premise and conclusion.
\end{definition}
Note that the flattening of a term depends on the choice of ``fresh'' variable in clause \ref{itm:flattening-composite}.
\todo{Explain more how we're always asusming that the variable is picked such that it does not occur in other syntactic objects in the current scope.}

\begin{lemma}
  \label{lem:flattening-versus-interpretation}
  Let $R = P \sqcup F$ be the structure of an algebraic signature and let $X$ be an algebraic structure.
  \begin{enumerate}
    \item
      Let $t$ be a term and let $\mathcal{I}$ be an interpretation of the variables occuring in $t$ in $X$.
      Then $\mathcal{I}$ is can be extended to the term $t$ if and only if $\mathcal{I}$ can be extended to an interpretation of of the relational formula $\mathrm{Flat}(t)$.
      In either case, if  an $\mathcal{J}$ exists, then it exists uniquely, and $\mathcal{J}(v_\mathrm{Flat}(t)) = \mathcal{I}(t)$.
    \item
      Let $\phi$ be an algebraic atom and let $\mathcal{I}$ be an interpretation of the variables occuring in $\phi$ in $X$.
      Then $\mathcal{I}$ is an interpretation of $\phi$ if and only if $\mathcal{I}$ can be extended to an interpretation $\mathcal{J}$ of the relational formula $\mathrm{Flat}(\phi)$.
      If $\mathcal{J}$ exists, then it exists uniquely.
    \item
      Let $\mathcal{S}$ be an algebraic sequent.
      Then $X$ satisfies $\mathcal{S}$ if and only if it satisfies the relational sequent $\mathrm{Flat}(\mathcal{S})$.
  \end{enumerate}
\end{lemma}
\begin{proof}
  By construction.
\end{proof}

\begin{theorem}
  Let $R = P \sqcup F$ be the structure of an algebraic signature, let $T$ be a set of algebraic sequents, and let $X$ be a relational structure.
  Then $X$ is an algebraic structure that satisfies the algbraic sequents in $T$ if and only if it satisfies the sequents \eqref{eq:functionality-axiom} for all $f \in F$ and the flattenings of all sequents in $T$.
  \qed
\end{theorem}

\begin{lemma}
  \label{lem:unflatten}
  Let $\mathcal{S} = (S, P \sqcup F)$ be an algebraic signature.
  Let $\phi$ be a relational formula.
  Then there exists an algebraic formula $\mathrm{Unflat}(\phi)$ with the following properties:
  \begin{enumerate}
    \item
      Every variable occurs in $\phi$ if and only if it occurs in $\mathrm{Unflat}(\phi)$.
    \item
      Let $\mathcal{I}$ be an interpretation of the variables of $\phi$ in an algebraic structure $X$.
      Then $\mathcal{I}$ is an interpretation of the algebraic formula $\phi$ if and only if it is an interpretation of the relational formula $\mathrm{Unflat}(\phi)$.
  \end{enumerate}
\end{lemma}
\begin{proof}
  Replace every relational atom of the form $f(x_1, \dots, x_n, x)$ for some function symbol $f : s_1 \times \dots \times s_n \rightarrow s$ with the algebraic atom $f(x_1, \dots, x_n, x)$.
\end{proof}

\todo{
  Add a converse:
  Every relational sequent can be obtained as flattening of an algebraic sequent.
}

\subsection{Satisfaction as lifting problem}

\begin{definition}
  We associate to each flat atom $\phi$ a relational structure $[\phi]$ and an interpretation $\mathcal{I}_\phi$ of $\phi$ in $[\phi]$ as follows:
  \begin{enumerate}
    \item
      If $\phi = r(v_1, \dots, v_n)$ where $r : s_1 \times \dots \times s_n$, then the carriers of $[\phi]$ are given by distinct elements $\mathcal{I}_\phi(v_i) \in [\phi]_{s_i}$, and $(\mathcal{I}_\phi(v_1), \dots, \mathcal{I}_\phi(v_n)) \in r_{[\phi]}$.
      The relations $r'_{[\phi]}$ for $r \neq r'$ are empty.
    \item
      If $\phi = v \downarrow$, where $v$ has sort $s$, then $[\phi]_s$ contains a single element $\mathcal{I}_\phi(v)$.
      All other carrier sets and all relations are empty.
    \item
      If $\phi = v_1 \equiv v_2$, where $v_1$ and $v_2$ have sort $s$, then $[\phi]_s$ contains a single element $\mathcal{I}_\phi(v_1) = \mathcal{I}_\phi(v_2)$.
      All other carrier sets and all relations are empty.
  \end{enumerate}

  Let $\mathcal{F} = \phi_1 \land \dots \land \phi_n$ be a relational formula.
  Then we denote by $[\mathcal{F}]$ the relational structure
  \begin{equation}
    ([\phi_1] \amalg \dots \amalg [\phi_n]) / \sim
  \end{equation}
  where $\sim$ is the relation given by
  \begin{equation}
    \mathcal{I}_{\phi_i}(v) \sim \mathcal{I}_{\phi_j}(v)
  \end{equation}
  for all $i, j \in \{1, \dots, n\}$ and variables $v$ occuring in both $\phi_i$ and $\phi_j$.
  
  Let $\mathcal{S} = \mathcal{F} \Rightarrow \mathcal{G}$ be a relational sequent.
  Then we denote by $[\mathcal{S}]$ the evident map $[\mathcal{F}] \rightarrow [\mathcal{F} \land \mathcal{G}]$.
\end{definition}

\begin{proposition}
  \label{prop:relational-formula-structure-universal}
  Let $\mathcal{F}$ be a relational formula and let $X$ be a relational structure.
  Then there is a bijection between interpretations of $\mathcal{F}$ in $X$ and maps $[\mathcal{F}] \rightarrow X$.
\end{proposition}
\begin{proof}
  If $f : [\mathcal{F}] \rightarrow X$, then $f \circ I_\mathcal{F}$ is an interpretation of $\mathcal{F}$ in $X$.
  For the converse, let $\mathcal{F} = \phi_1 \land \dots \land \phi_n$ for relational atoms $\phi_i$.
  Then every interpretation $\mathcal{I}$ of $\mathcal{F}$ restricts to an interpretation of $\phi_i$ for each $i$.
  The carrier of $[\phi_i]$ is defined using the variables of $\phi_i$, which defines an evident map $[\phi_i] \rightarrow X$.
  Since the restrictions of $\mathcal{I}$ to the variables in each $\phi_i$ agree on variables that occur in simultaneously in two atoms, the individual maps $[\phi_i] \rightarrow X$ glue to a map $[\mathcal{F}] \rightarrow X$.
\end{proof}

In light of this universal property, we conflate relational formulas $\mathcal{F}$ with the relational structure $[\mathcal{F}]$ and variables occuring in $\mathcal{F}$ with elements of $[\mathcal{F}]$ when confusion is unlikely.
Similarly, we conflate relational sequents $\mathcal{S}$ with the corresponding morphisms $[\mathcal{S}]$.
\todo{
  Does that actually happen somewhere?
}

\begin{proposition}
  \label{prop:relational-sequent-morphism-lifts}
  Let $\mathcal{S}$ be a relational sequent and let $X$ be a relational structure.
  Then $X$ satisfies $\mathcal{S}$ if and only if $X$ is injective to $[\mathcal{S}]$.
\end{proposition}
\begin{proof}
  Let $\mathcal{S} = \mathcal{F} \Rightarrow \mathcal{G}$.
  By Proposition \ref{prop:relational-formula-structure-universal}, interpretations $\mathcal{I}$ of the premise $\mathcal{F}$ correspond to maps $\langle \mathcal{I} \rangle : [\mathcal{F}] \rightarrow X$, and interpretations $\mathcal{J}$ of $\mathcal{F} \land \mathcal{G}$ correspond to maps $\langle \mathcal{J} \rangle : [\mathcal{F} \land \mathcal{G}] \rightarrow X$.
  The map $[\mathcal{S}] : [\mathcal{F}] \rightarrow [\mathcal{F} \land \mathcal{G}]$ can be obtained by restriction of the generic interpretation of $\mathcal{F} \land \mathcal{G}$ in $[\mathcal{F} \land \mathcal{G}]$ to the variables occurring in $\mathcal{F}$.
  It follows that
  \begin{equation}
    \begin{tikzcd}
      \left[\mathcal{F}\right] \arrow[r, "\langle \mathcal{I} \rangle"] \arrow[d, "{[\mathcal{S}]}"'] & X \\
      \left[\mathcal{F} \land \mathcal{G}\right] \arrow[ur, "\langle \mathcal{J} \rangle"']
    \end{tikzcd}
  \end{equation}
  commutes if and only if $\mathcal{I}$ is a restriction of $\mathcal{J}$.
\end{proof}

\begin{definition}
  Let $\mathcal{S} = (S, P \sqcup F)$ be an algebraic signature.
  We associate to each algebraic formula $\mathcal{F}$ the algebraic structure
  \begin{equation}
    [\mathcal{F}] = \mathrm{FAlg}([\mathrm{Flat}(\mathcal{F}))
  \end{equation}
  and to each algebraic sequent $\mathcal{S} = \mathcal{F} \Rightarrow \mathcal{G}$ the evident morphism
  \begin{equation}
    [\mathcal{S}] : [\mathcal{F}] \rightarrow [\mathcal{F} \land \mathcal{G}]
  \end{equation}
  of algebraic structures.
\end{definition}

\begin{proposition}
  \label{prop:algebraic-formula-structure-universal}
  Let $\mathcal{S} = (S, P \sqcup F)$ be an algebraic signature.
  Let $\mathcal{F}$ be an algebraic formula and let $X$ be an algebraic structure.
  Then there is a bijection between interpretations of $\mathcal{F}$ in $X$ and maps $[\mathcal{F}] \rightarrow X$.
\end{proposition}
\begin{proof}
  This follows by combining Lemma \ref{lem:flattening-versus-interpretation}, Proposition \ref{prop:algebraic-formula-structure-universal} and the universal property of the free algebraic structure functor.
\end{proof}

\begin{proposition}
  Let $\mathcal{S}$ be an algebraic sequent and let $X$ be an algebraic structure.
  Then $X$ satisfies $\mathcal{S}$ if and only if $X$ is injective to $[\mathcal{S}]$.
\end{proposition}
\begin{proof}
  Analogously to the proof of Proposition \ref{prop:relational-sequent-morphism-lifts}.
\end{proof}

\begin{lemma}
  \label{lem:composition-sequent-morphisms}
  Let $\mathcal{S} = \mathcal{F} \Rightarrow \mathcal{G}$ and $\mathcal{T} = \mathcal{G} \Rightarrow \mathcal{H}$ be relational (algebraic) sequents.
  Then
  \begin{equation}
    [\mathcal{F} \land \mathcal{G} \Rightarrow \mathcal{H}] \circ [\mathcal{F} \Rightarrow \mathcal{G}] = [\mathcal{F} \Rightarrow \mathcal{G} \land \mathcal{H}].
  \end{equation}
\end{lemma}
\begin{proof}
  Follows from Propositions \ref{prop:relational-formula-structure-universal} and \ref{prop:algebraic-formula-structure-universal}.
\end{proof}

\subsection{Epimorphisms of relational structures}
\todo{
  Kind of a bad name for this section, since we also classify in general the image of $\mathcal{S} \mapsto [\mathcal{S}]$, which includes not only epimorphisms.
}

\begin{definition}
  Let $f : X \rightarrow Y$ be a map of relational structures.
  We say that $f$ is \emph{surjective} if $f_s: X_s \rightarrow Y_s$ is a surjective map for all $s \in S$.
\end{definition}

\begin{proposition}
  \label{prop:epimorphisms-classification}
  Let $\mathcal{S} = (S, R)$ be a relational signature.
  Let $f : X \rightarrow Y$ be a map of relational structures.
  \begin{enumerate}
    \item
      \label{itm:relational-epimorphisms}
      $f$ is an epimorphism if and only if $f$ is surjective.
    \item
      \label{itm:relational-effective-epimorphisms}
      $f$ is an effective epimorphism if and only if $f$ is surjective and furthermore the induced maps $f_r : r_X \rightarrow r_Y$ are surjective for all $r \in R$.
  \end{enumerate}
\end{proposition}
\begin{proof}
  \ref{itm:relational-epimorphisms}.
  Note that, for every morphism $f : X \rightarrow Y$ in a cocomplete category $\mathcal{C}$, $f$ is an epimorphism if and only if
  \begin{equation}
    \begin{tikzcd}
      A \arrow[r, "f"] \arrow[d, "f"'] & A \arrow[d] \\
      A \arrow[r] & A
    \end{tikzcd}
  \end{equation}
  is a pushout square.
  Because the forgetful functor from relational structures to $S$-indexed families of sets preserves colimits, it follows that it preserves epimorphisms, i.e. that every epimorphism of relational structures must be surjective.
  The same forgetful functor is full and faithful, hence reflects epimorphisms.
  This shows that surjective maps of relational structures are indeed epimorphisms.

  \ref{itm:relational-effective-epimorphisms}.
  This follows from the construction of colimits in Proposition \ref{prop:rel-cocomplete}.
\end{proof}

\begin{definition}
  Let $\mathcal{S}$ be a relational sequent.
  We say that $\mathcal{S}$ is \emph{surjective} if every variable in the conclusion of $\mathcal{S}$ also appears in the premise.
  We say that $\mathcal{S}$ is \emph{effectively epi} if every atom in the conclusion of $\mathcal{S}$ is of the form $v_1 \equiv v_2$ for variables $v_1, v_2$ that also occur in the premise.
\end{definition}

\begin{proposition}
  \label{prop:semantics-relational-sequents}
  The image of (surjective, effectively epi) relational sequents under the assignment $\mathcal{S} \mapsto [\mathcal{S}]$ can be described as follows:
  \begin{enumerate}
    \item
      \label{itm:semantics-arbitrary-relational-sequent}
      If $\mathcal{S}$ is a relational sequent, then $[\mathcal{S}]$ is a map finite relational structures.
      Conversely, every map of finite relational structures is isomorphic to a map of the form $[\mathcal{S}]$ for a relational sequent $\mathcal{S}$.
    \item
      \label{itm:semantics-surjective-relational-sequent}
      If $\mathcal{S}$ is surjective, then $[\mathcal{S}]$ is a surjection of finite relational structures.
      Conversely, every surjection of finite relational structures is isomorphic to a map of the form $[\mathcal{S}]$ for a surjective relational sequent $\mathcal{S}$.
    \item
      \label{itm:semantics-quotient-relational-sequent}
      If $\mathcal{S}$ is effectively epi, then $[\mathcal{S}]$ is an effective epimorphism of finite relational structures.
      Conversely, every effective epimorphism of finite relational structures is isomorphic to a map of the form $[\mathcal{S}]$ for an effectively epi relational sequent $\mathcal{S}$.
  \end{enumerate}
\end{proposition}
\begin{proof}
  \ref{itm:semantics-arbitrary-relational-sequent}.
  Relational sequents $\mathcal{S}$ are by definition finite formulas, so it follows from the definition of the map $[\mathcal{S}]$ that its domain and codomain are finite.

  Conversely let $f : A \rightarrow B$ be a map of finite relational structures.
  Choose distinct variables $v_x$ of sort $s$ for all sorts $s$ and $x \in A_s$.
  The formulas
  \begin{mathpar}
    \mathcal{F}_{\mathrm{car}} = \bigwedge_{x \in A} v_x \downarrow
    \and
    \mathcal{F}_{\mathrm{rel}} = \bigwedge_{\substack{r \in R\\(x_1, \dots, x_n) \in r_A}} r(v_{x_1}, \dots, v_{x_n})
  \end{mathpar}
  are finite (and hence well-defined) because $A$ is finite.
  The formula $\mathcal{F}_{\mathrm{car}}$ encodes the carriers of $A$, and $\mathcal{F}_{\mathrm{rel}}$ encodes the relations.
  Thus if $\mathcal{F} = \mathcal{F}_{\mathrm{car}} \land \mathcal{F}_{\mathrm{rel}}$, then $A \cong [\mathcal{F}]$.

  Define similarly variables $v_y$ for each $y \in B$ and formulas $\mathcal{G}_{\mathrm{car}}, \mathcal{G}_{\mathrm{rel}}$ for $B$.
  Set
  \begin{equation}
    \mathcal{G}_\mathrm{eq} = \bigwedge_{x \in A} v_x \equiv v_{f(x)}
  \end{equation}
  and let $\mathcal{G} = \mathcal{G}_{\mathrm{car}} \land \mathcal{G}_{\mathrm{rel}} \land \mathcal{G}_{\mathrm{eq}}$.
  Then
  \begin{equation}
    B \cong [\mathcal{G}_{\mathrm{car}} \land \mathcal{G}_{\mathrm{rel}}] \cong [\mathcal{G}] \cong [\mathcal{G} \land \mathcal{F}]
  \end{equation}
  and it can be verified using the universal property of $[\mathcal{F}]$ (Proposition \ref{prop:relational-formula-structure-universal}) that
  \begin{equation}
    \begin{tikzcd}
      \left[\mathcal{F}\right] \arrow[r] \arrow[d, "\cong"'] & \left[\mathcal{F} \land \mathcal{G} \right] \arrow[d, "\cong"] \\
      A \arrow[r, "f"] & B
    \end{tikzcd}
  \end{equation}
  commutes.

  \ref{itm:semantics-surjective-relational-sequent}.
  The carrier of a relational structure $[\mathcal{F}]$ obtained from a formula $\mathcal{F}$ is a quotient of the set of variables occuring in $\mathcal{F}$.
  Thus if every variable in the conclusion $\mathcal{G}$ of a sequent $\mathcal{F} \Rightarrow \mathcal{G}$ occurs also in the premise $\mathcal{F}$, then the resulting map $[\mathcal{F} \Rightarrow \mathcal{G}]: [\mathcal{F}] \rightarrow [\mathcal{F} \land \mathcal{G}]$ is surjective.
  Conversely, let $f : A \rightarrow B$ be a surjective map of finite relational structures.
  Let $v_x$ for $x \in A$ and $\mathcal{F} = \mathcal{F}_\mathrm{car} \land \mathcal{F}_\mathrm{rel}$ be as in the proof of \ref{itm:semantics-arbitrary-relational-sequent}, so that $[\mathcal{F}] \cong A$.
  Set
  \begin{mathpar}
    \mathcal{G}_\mathrm{eq} = \bigwedge_{\substack{x, y \in A\\f(x) = f(y)}} v_x \equiv v_y
    \and
    \mathcal{G}_{\mathrm{rel}} = \bigwedge_{\substack{r \in R\\x_1, \dots, x_n \in A\\(f(x_1), \dots, f(x_n)) \in r_B}} r(v_{x_1}, \dots, v_{x_n})
  \end{mathpar}
  and $\mathcal{G} = \mathcal{G}_\mathrm{eq} \land \mathcal{G}_\mathrm{rel}$.
  Because all the the $v_x$ occur in $\mathcal{F}$, all the variables of $\mathcal{G}$ occur in $\mathcal{F}$.
  By definition of $\mathcal{G}_\mathrm{eq}$, the map $[\mathcal{F} \land \mathcal{G}_\mathrm{eq}] \rightarrow B$ is an isomorphism on carriers, which then implies that $[\mathcal{F} \land \mathcal{G}] \cong B$ by definition of $\mathcal{G}_\mathrm{rel}$.
  Thus $[\mathcal{\mathcal{F}} \Rightarrow \mathcal{G}] \cong f$.

 \ref{itm:semantics-quotient-relational-sequent}.
  If $\mathcal{S}$ is effectively epi and $f = [\mathcal{S}] : A \rightarrow B$, then the maps $f_r : r_A \rightarrow r_B$ are surjective for all relation symbols $r$.
  Thus by Proposition \ref{prop:epimorphisms-classification}, $f$ is an effective epimorphism.

  
  Conversely, let $f : A \rightarrow B$ be an effective epimorphism of finite relational structures.
  Let $v_x$ for $x \in A$ and $\mathcal{F} = \mathcal{F}_\mathrm{car} \land \mathcal{F}_\mathrm{rel}$ be as in the proof of \ref{itm:semantics-arbitrary-relational-sequent}, so that $[\mathcal{F}] \cong A$.
  Let
  \begin{equation}
    \mathcal{G} = \mathcal{G}_\mathrm{eq} = \bigwedge_{\substack{x, y \in A\\f(x) = f(y)}} v_x \equiv v_y.
  \end{equation}
  By the same argument as in the proof of \ref{itm:semantics-surjective-relational-sequent}, it follows that $[\mathcal{F} \land \mathcal{G}] \rightarrow B$ is an isomorphism on carriers.
  Let $r$ be a relation symbol and let $(y_1, \dots, y_n) \in r_B$.
  Then because $f$ is an effective epimorphism, there exist $x_1, \dots, x_n \in A$ such $f(x_i) = y_i$ for all $i$ and $(x_1, \dots, x_n) \in r_A$.
  Thus $\mathcal{F}$ contains the atom $r(v_{x_1}, \dots, v_{x_n})$, hence $(y_1, \dots, y_n)$ is in the image of $r_{[\mathcal{F} \land \mathcal{G}]} \rightarrow B$.
  We conclude $[\mathcal{F} \land \mathcal{G}] \cong B$, hence $f \cong [\mathcal{F} \Rightarrow \mathcal{G}]$. 
\end{proof}

\begin{corollary}
  Let $T$ be a set of surjective relational sequents.
  Then $[T] = \{ [\mathcal{S}] \mid \mathcal{S} \in T\}$ is flub (Definition \ref{def:flub}).
  The category $[T]^\pitchfork$ is a reflective subcategory of the category of $\mathrm{Rel}$.
\end{corollary}
\begin{proof}
  Every morphism $[\mathcal{S}]$ for $\mathcal{S} \in T$ is a surjection, hence an epimorphism.
  Thus lifts against $[\mathcal{S}]$ exist uniquely if they exist, and Proposition \ref{prop:orth-refl-constr-exist} applies.
\end{proof}

\todo{
  TODO: Some remark/prose about this being the basis of eqlog.
}

\subsection{Epimorphisms of algebraic structures}
\todo{
  Bad name for the same reason as analogous subsection for relational structures.
}
Let $\mathcal{S} = (S, P \sqcup F)$ be an algebraic signature.

\begin{definition}
  Let $\mathcal{S}$ be an algebraic sequent.
  We say that $\mathcal{S}$ is \emph{epi} if every variable in the conclusion of $\mathcal{S}$ also occurs in the premise.
  We say that $\mathcal{S}$ is \emph{effectively epi} if every atom in the conclusion of $\mathcal{S}$ is of the form $t_1 \equiv t_2$ for terms $t_1, t_2$ that also occur in the premise.
\end{definition}

\begin{definition}
  Let $f$ be a function symbol.
  The \emph{totality sequent} $f \downarrow$ is given by
  \begin{equation}
    v_1 \downarrow \land \dots \land v_n \downarrow \Rightarrow f(v_1, \dots, v_n)\downarrow.
  \end{equation}
\end{definition}

\begin{proposition}
  Let $f$ be a function symbol. 
  \begin{enumerate}
    \item
      An algebraic structure $X$ satisfies $f \downarrow$ if and only if $f_X$ is a total function.
    \item
      $[f \downarrow]$ is an epimorphism of algebraic structures.
      \qed
  \end{enumerate}
\end{proposition}

\begin{definition}
  Let $g : X \rightarrow Y$ be  a map of algebraic structures.
  We say that $X$ is \emph{saturated} (with respect to $g$) if and only if for all function symbols $f$ and elements $x_1, \dots, x_n \in X$, the implication
  \begin{equation}
    f_Y(g(x_1), \dots, g(x_n)) \downarrow \implies f_Y(x_1, \dots, x_n) \downarrow
  \end{equation}
  holds.
\end{definition}

\begin{proposition}
  \label{prop:saturation}
  Let $\mathrm{Tot} = \{ [f \downarrow] \mid f \in F \}$, and let $g : X \rightarrow Y$ be a map of algebraic structures.
  Then there exists a factorization
  \begin{equation}
    \begin{tikzcd}
      X \arrow[dr, "g"'] \arrow[rr, "h"] & & X' \arrow[dl, "g'"] \\
      & Y
    \end{tikzcd}
  \end{equation}
  such that $h$ is a relative $\mathrm{Tot}$-cell complex and $g'$ is saturated.
  Moreover, the triple $(X', h, g')$ is unique up to unique isomorphism.
\end{proposition}
\begin{proof}
  Consider the set $M$ of all triples $(f, a, b)$ commuting triangles
  \begin{equation}
    \begin{tikzcd}
      \cdot \arrow[dr, "a"'] \arrow[rr, "{[f \downarrow]}"] & & \cdot \arrow[dl, "b"] \\
      & Y
    \end{tikzcd}
  \end{equation}
  $M$ is a set of epimorphisms in the slice category $\mathrm{Alg}_{/ Y}$.
  It follows that $M$ is flub, so $M^\pitchfork$ is a reflective subcategory of $\mathrm{Alg}_{/ Y}$.
  
  A map with codomain $Y$ is saturated if and only if, as an object of $\mathrm{Alg}_{/ Y}$, it is injective to $M$.
  Thus the reflection of $g$ into $M^\pitchfork$ results into a factorization $g = g' \circ h$ where $h$ is a relative $M$-cell complex and $g'$ is saturated.
  Because the forgetful functor $\mathrm{Alg}_{/ Y} \rightarrow \mathrm{Alg}$ preserves colimits, $h$ is also a relative $\mathrm{Tot}$-cell complex.
  \todo{
    Also show that $h$ is a relative $\mathrm{Tot}$-cell complex, then it is a relative $M$-cell complex?
    That's how you show uniqueness.
  }
\end{proof}

\begin{definition}
  Let $g : X \rightarrow Y$ be a map of algebraic structures.
  We call the unique factorization $g = g' h$ as in Proposition \ref{prop:saturation} the \emph{saturization} of $g$.
\end{definition}

\begin{lemma}
  \label{lem:saturated-epis}
  Let $g : X \rightarrow Y$ be a saturated map of algebraic structures.
  Then $g$ is an epimorphism in $\mathrm{Alg}$ if and only if it is an epimorphism in $\mathrm{Rel}$.
\end{lemma}
\begin{proof}
  The free algebraic structure functor $\mathrm{Rel} \rightarrow \mathrm{Alg}$ preserves epimorphisms, which shows necessity of the condition.

  Note that $\operatorname{Im}_\mathrm{Alg} g = \operatorname{Im}_\mathrm{Rel} g$, that is, the image algebraic structure can be computed as image of relational structures.
  Because $g$ is an epimorphism in $\mathrm{Rel}$ or $\mathrm{Alg}$ if and only if the inclusion of the image in the corresponding category is an epimorphism, we may assume that $g$ is a monomorphism, i.e. injective on carriers.

  Consider the the pushout $Z = Y \amalg^\mathrm{Rel}_X Y$ in $\mathrm{Rel}$.
  There are inclusions $Y \cong Y_0 \subseteq Z$, $Y \cong Y_1 \subseteq Z$ corresponding to the two components of $Z$ such that $Y_0 \cup Y_1 = Z$, and an inclusions $X \subseteq Y_0, X \subseteq Y_1$.
  We have $(Y_0)_s \cap (Y_1)_s = X_s$ for all sorts $s$, but note that the analogous equation does not hold for relations:
  Epimorphisms of relational structures need not be surjective on relations.

  We claim that $Z$ is an algebraic structure.
  Thus let $f$ be a function symbol, and let $z_1, \dots, z_n, z, z' \in Z$ such that $\bar z = (z_1, \dots, z_n, z) \in f_Z$ and $\bar z' = (z_1, \dots, z_n, z') \in f_Z$.
  We need to show that $z = z'$.

  If $\bar z, \bar z' \in Y_0$ or $\bar z, \bar z' \in Y_1$ this follows from the fact that $Y_0 \cong Y \cong Y_1$ is an algebraic structure.
  We may thus (by symmetry) assume that $\bar z \in Y_0$ and $\bar z' \in Y_1$.
  Now $z_1, \dots, z_n \in Y_0 \cap Y_1$, hence $z_X$.
  Because the inclusions $X \subseteq Y_i$ are saturated, we have $f_X(z_1, \dots, z_n) = z''$ for some $z'' \in X$.
  It follows that $z = z''$ and $z' = z''$, hence $z = z'$.

  Now, if $g$ is an epimorphism in $\mathrm{Alg}$, then the two maps $Y \rightarrow Y \amalg^\mathrm{Alg}_X Y = Z'$ agree.
  But we have just shown that $Z' = Z$, so also the two maps $Y \rightarrow Y \amalg^\mathrm{Rel}_X Y$ agree, which implies that $g$ is surjective, i.e. an epimorphism in $\mathrm{Rel}$.
\end{proof}

\begin{proposition}
  \label{prop:algebraic-epis}
  Let $g : X \rightarrow Y$ be a map of algebraic structures.
  \begin{enumerate}
    \item
      \label{itm:algebraic-epimorphisms}
      Let $g = g' h$ be the totalization of $g$.
      Then $g$ is an epimorphism in $\mathrm{Alg}$ if and only if $g'$ is an epimorphism in $\mathrm{Rel}$.
    \item
      \label{itm:algebraic-effective-epimorphisms}
      $g$ is an effective epimorphism in $\mathrm{Alg}$ if and only if it is an effective epimorphism in $\mathrm{Rel}$.
  \end{enumerate}
\end{proposition}
\begin{proof}
  \ref{itm:algebraic-epimorphisms}.
  Every morphism in $\mathrm{Tot}$ is an epimorphism.
  Since epimorphisms are stable under pushouts and transfinite compositions, every relative $\mathrm{Tot}$-cell and in particular $h$ is an effective epimorphism.
  We conclude with Lemma \ref{lem:saturated-epis}.

  \ref{itm:algebraic-effective-epimorphisms}.
  This follows from general properties of colimits and reflective subcategories.
\end{proof}

\begin{definition}
  An algebraic sequent $\mathcal{S}$ is called \emph{epi} if every variable that occurs in the conclusion of $\mathcal{S}$ also occurs in the premise.
\end{definition}

\begin{proposition}
  Let $\mathcal{S}$ be an algebraic sequent.
  If $\mathcal{S}$ is epi, then the reflection of $[\mathrm{Flat}(\mathcal{S}]$ into the category of algebraic structures is an epimorphism of finite algebraic structures.
  Conversely, if $f : A \rightarrow B$ is an epimorphism of finite algebraic structures, then there exists an epi algebraic sequent $\mathcal{S}$ such that $f \cong [\mathcal{S}]$.
  \todo{Phrase this like the statement about relational sequents/structures.}
  \todo{Also include something on arbitrary maps of finite relational structures?}
\end{proposition}

\begin{lemma}
  \todo{This belongs in a previous section.}
  Let $M$ be flub set of morphisms, and let $f : A \rightarrow B$.
  Denote by $\overline{f} : \overline{A} \rightarrow \overline{B}$ the reflection of $f$ into $M^\pitchfork$.
  Then
  \begin{equation}
    (M \cup \{ f \})^\perp = (M \cup \{ \overline{f} \})^\perp
  \end{equation}
  and
  \begin{equation}
    (M \cup \{ f \})^\pitchfork = (M \cup \{ \overline{f} \})^\pitchfork.
  \end{equation}
\end{lemma}
\begin{proof}
  Let $X$ be orthogonal (equivalently, injective) to $M$.
  Then there is bijective correspondance between solutions to the following lifting problems:
  \begin{equation}
    \begin{tikzcd}    
      A \arrow[r, "a"] \arrow[d, "f"'] & X \\
      B \arrow[ur, dashed]
    \end{tikzcd}    
    \qquad
    \begin{tikzcd}    
      \overline{A} \arrow[r, "a'"] \arrow[d, "\overline{f}"'] & X \\
      \overline{B}. \arrow[ur, dashed]
    \end{tikzcd}    
  \end{equation}
  Here $a$ is an arbitrary map, and $a' : \overline{A} \rightarrow X$ is induced from $a$ by the universal property of $\overline{A}$ and $X$ being orthogonal to $\overline{f}$.
\end{proof}

\begin{lemma}
  \label{lem:flattening-atoms}
  Let $\mathcal{S} = (S, P \sqcup F)$ be an algebraic signature.
  Let $\mathcal{F}$ be an algebraic formula and let $V$ be the set of variables that occur in $\mathcal{F}$.
  Let $\mathrm{Flat}(\mathcal{F}) = \phi_1 \land \dots \land \phi_n$ for relational atoms $\phi_i$.
  Let $\phi = \phi_i$ for some $i$ and let
  \begin{equation}
    V = \{ v \mid v \text{ occurs in } \mathcal{F} \text{ or in } \phi_j \text{ for some } j < i \}.
  \end{equation}
  Then $\phi$ has one of the following forms:
  \begin{enumerate}
    \item
      \label{itm:flattened-predicate-case}
      $\phi = p(v_1, \dots, v_m)$, where $p$ is a predicate symbol and $v_1, \dots, v_m \in V$.
    \item
      \label{itm:flattened-function-case}
      $\phi = f(v_1, \dots, v_m, v)$, where $f : s_1 \times \dots \times s_m \rightarrow s$ is a function symbol, $v_1, \dots, v_m \in V$ and $v \notin V$.
    \item
      \label{itm:flattened-defined-case}
      $\phi = v \downarrow$, where $v \in V$.
    \item
      \label{itm:flattened-equals-case}
      $\phi = v_1 \equiv v_2$, where $v_1, v_2 \in V$.
  \end{enumerate}
\end{lemma}
\begin{proof}
  \todo{Tedious induction.}
\end{proof}

\begin{proposition}
  \label{prop:semantics-algebraic-sequents}
  Let $\mathcal{S} = (S, P \sqcup F)$ be an algebraic signature.
  The image of (epi, quotient) algebraic sequents under the assignment $\mathcal{S} \mapsto [\mathcal{S}]$ can be described as follows:
  \begin{enumerate}
    \item
      \label{itm:semantics-arbitrary-algebraic-sequent}
      If $\mathcal{S}$ is an algebraic sequent, then $[\mathcal{S}]$ is a map of finite algebraic structures.
      Conversely, every map of finite relational structures is isomorphic to a map of the form $[\mathcal{S}]$ for some algebraic sequent $\mathcal{S}$.
    \item
      \label{itm:semantics-epi-algebraic-sequent}
      If $\mathcal{S}$ is an epi algebraic sequent, then $[\mathcal{S}]$ is an epimorphism of finite algebraic structures.
      Conversely, every epimorphism of finite relational structures is isomorphic to a map of the form $[\mathcal{S}]$ for some epi algebraic sequent $\mathcal{S}$.
    \item
      \label{itm:semantics-quotient-algebraic-sequent}
      If $\mathcal{S}$ is a quotient algebraic sequent, then $[\mathcal{S}]$ is an effective epimorphism of finite algebraic structures.
      Conversely, every effective epimorphism of finite relational structures is isomorphic to a map of the form $[\mathcal{S}]$ for some quotient algebraic sequent $\mathcal{S}$.
  \end{enumerate}
\end{proposition}
\begin{proof}
  \ref{itm:semantics-arbitrary-algebraic-sequent}.
  If $\mathcal{S}$ is an algebraic sequent, then by \ref{prop:semantics-relational-sequents}, $[\mathrm{Flat}(\mathcal{S})]$ is a map of finite relational structures, hence $\mathrm{FAlg}([\mathrm{Flat}(\mathcal{S})])$ is a map of finite algebraic structures.

  Conversely, let $g : X \rightarrow Y$ be a map of finite algebraic structures.
  Then $g$ is, in particular, a map of finite relational structures.
  By \ref{prop:semantics-relational-sequents}, there exists a relational sequent $\mathcal{S}$ such that $[\mathcal{S}] \cong g$.
  We can now invoke Lemma \ref{lem:unflatten} to obtain an algebraic sequent $\mathrm{Unflat}(\mathcal{S})$ such that $[\mathrm{Unflat}(\mathcal{S})] \cong g$.

  \ref{itm:semantics-quotient-algebraic-sequent}.
  Let $\mathcal{S} = \mathcal{F} \Rightarrow \mathcal{G}$ be a quotient relational sequent.
  By Lemma \ref{lem:composition-sequent-morphisms}, we may assume that $\mathcal{G} = t_1 \equiv t_2$ for terms $t_1, t_2$ that also occur in the premise.
  It follows that $v_\mathrm{Flat}(t_1)$ and $v_\mathrm{Flat}(t_2)$ occur in $\mathrm{Flat}(\mathcal{F})$.
  Thus $\mathcal{S}' = \mathrm{Flat}(\mathcal{F})) \implies v_\mathrm{Flat}(t_1) \equiv v_\mathrm{Flat}(t_2)$ is a quotient relational sequent such that $[\mathcal{S}] = \mathrm{FAlg}([\mathcal{S}'])$.
  By clause \ref{itm:semantics-quotient-relational-sequent} of Proposition \ref{prop:semantics-relational-sequents}, $[\mathcal{S}']$ is an effective epimorphism of relational structures.
  Because the free algebra functor is a left adjoint, it preserves effective epimorphisms.
  Thus $[\mathcal{S}]$ is an effective epimorphism.

  \ref{itm:semantics-epi-algebraic-sequent}.
  Let $\mathrm{Flat}(\mathcal{S}) = \mathcal{F} \Rightarrow \mathcal{G}$ be the flattening of an epi algebraic formula.
  We need to show that $g = \mathrm{FAlg}([\mathcal{F} \Rightarrow \mathcal{G}])$ is an epimorphism in $\mathrm{Alg}$.

  By Lemma \ref{lem:composition-sequent-morphisms}, we may without loss of generality assume that $\mathcal{G} = \mathcal{\phi}$ is a relational atom of one of the types as in Lemma \ref{lem:flattening-atoms}, where $\phi$ is the set of variables in $\mathcal{F}$.
  In cases \ref{itm:flattened-predicate-case}, \ref{itm:flattened-defined-case} and \ref{itm:flattened-equals-case}, the map $[\mathcal{F} \Rightarrow \phi]$ is surjective. 
  Because the free algebra functor preserves epimorphisms, this implies that $g$ is an epimorphism in $\mathrm{Alg}$.
  In case \ref{itm:flattened-predicate-case}, $[\mathcal{F} \Rightarrow \phi]$ is a pushout of $[f \downarrow]$ in $\mathrm{Rel}$.
  Since $[f \downarrow] = \mathrm{FAlg}([f \downarrow])$ is an epimorphism in $\mathrm{Alg}$ and pushouts preserve epimorphisms, it follows that also in this case $g$ is an epimorphism.

  Conversely, let $g : X \rightarrow Y$ be an epimorphism of finite algebraic structures.
  Let $g = g' h$ be the saturation of $g$.
  Then $h$ is a relative $\mathrm{Tot}$-complex, hence $h$ is the colimit of a chain
  \begin{equation}
    X = X_0 \xrightarrow{h_0} X_1 \xrightarrow{h_1} X_2 \xrightarrow{h_2} \dots
  \end{equation}
  where each $h_n$ is a pushout
  \begin{equation}
    \begin{tikzcd}
      \left[ v_1 \downarrow \land \dots \land v_n \downarrow \right] \arrow[r, "{[f_n \downarrow]}"] \arrow[d, "a"] & \left[ f_n(v_1, \dots, v_n) \downarrow \right] \arrow[d, "b"] \\
      X_n \arrow[r, "h_n"] & X_{n + 1}
    \end{tikzcd}
  \end{equation}
  of a totality sequent $f_n \downarrow$.

  We claim that $h_n \cong [\mathcal{S}]$ for some epi algebraic sequent $\mathcal{S}$ for all $n \geq 0$.
  To verify this, choose first an algebraic formula $\mathcal{F}$ such that $[\mathcal{F}] \cong X_n$.
  A formula $\mathcal{F}$ with property exists because $X_n$ is finite.
  The map $a$ corresponds to elements $x_1, \dots, x_n \in X$, and these elements correspond to terms $t_1, \dots, t_n$ that occur in $\phi$.
  Now $h_n \cong [\mathcal{F} \Rightarrow f_n(t_1, \dots, t_n)\downarrow]$.

  Let $g'_n : X_n \rightarrow \operatorname{colim}_i X_i \xrightarrow{h} Y$ for $n \geq 0$.
  Because $Y$ is finite, the chain
  \begin{equation}
    \operatorname{Im} g'_0 \subseteq \operatorname{Im} g'_1 \subseteq \dots \subseteq Y
  \end{equation}
  becomes stationary, say for $n \geq n_0$, and then $\operatorname{Im} g'_{n_0} = \operatorname{Im} g'$.
  $g$ is an epimorphism, hence $g'$ is a surjection by Proposition \todo{todo}, hence also $g'_{n_0} : X_{n_0} \rightarrow Y$ is a surjection.
  Thus $g'_{n_0} \cong [\mathcal{S}] \cong [\mathrm{Unflat}(\mathcal{S})]$ for some surjective relational sequent $\mathcal{S}$.
  Note that because $\mathcal{S}$ is surjective, the algebraic sequent $\mathrm{Unflat}(\mathcal{S})$ is epi.

  We have thus decomposed $g$ as
  \begin{equation}
    X = X_0 \xrightarrow{h_0} X_1 \xrightarrow{h_1} X_2 \rightarrow{h_2} \dots \xrightarrow{h_{n_0} - 1} X_{n_0} \xrightarrow{g_{n_0}} Y
  \end{equation}
  where each map is of the form $[\mathcal{S}]$ for some epi algebraic sequent.
  It follows that also $g = [\mathcal{S}]$ for some epi algebraic sequent $\mathcal{S}$.
\end{proof}

\subsection{Relational orthogonality classes as algebraic epi injectivity classes}

\begin{proposition}
  Let $\mathcal{S} = (S, R)$ be a relational signature.
  Let $M$ be a set of finite relational $\mathcal{S}$-structures.
  Then there exists an algebraic signature $\mathcal{S}' = (S, P \sqcup F)$ on the same set of sorts $S$ such that $P = R$ and a set $M'$ of epimorphisms of finite algebraic $\mathcal{S}'$-structures such that the forgetful functor $\mathrm{Alg}(\mathcal{S}') \rightarrow \mathrm{Rel}(\mathcal{S})$ restricts to an equivalence $M^\perp \simeq (M')^\pitchfork$.
\end{proposition}
\begin{proof}
  Choosing a relational seuqent $\mathcal{S}_g : \mathcal{F}_g \Rightarrow \mathcal{G}_g$ such that $[\mathcal{S}_g] \cong g$ for each $g \in M$, we identify $M$ with a set of relational sequents.

  Our set of function symbols $F$ is given by
  \begin{equation}
    F = \{ f_{\mathcal{S}, v} \mid \mathcal{S} : \mathcal{F} \Rightarrow \mathcal{G} \text{ is in } M \text{ and } v \text{ is a variable in } \mathcal{G} \}.
  \end{equation}
  Let $\mathcal{S} : A \Rightarrow B$ be in $M$ and let $v$ be a variable in $\mathcal{G}$.
  Let $v_1, \dots, v_n$ be an enumeration of the variables in $A$, let $s_i$ be the sort of $v_i$ and let $s$ be the sort of $v$.
  Then we define the signature of $f_{\mathcal{S}, v}$ by $f_{\mathcal{S}, v} : s_1 \times \dots \times s_n \rightarrow s$.

  Note that each relation symbol in $\mathcal{S}$ corresponds to a predicate symbol in $\mathcal{S}'$.
  We thus implicitly coerce relational $\mathcal{S}$-sequents to algebraic $\mathcal{S}'$-sequents (without invoking $\mathrm{Unflat}$).

  Let $M'$ be the set containing the following algebraic sequents, for all $\mathcal{F} \Rightarrow \mathcal{G}$ in $M$:
  \begin{enumerate}
    \item
      \label{itm:new-functions-satisfy}
      The sequent $\mathcal{F} \Rightarrow \mathcal{G}'$, where $\mathcal{G}'$ is obtained from $\mathcal{G}$ by replacing each variable $v$ in $\mathcal{G}$ with $f_{\mathcal{S}, v}(v_1, \dots, v_n)$.
    \item
      \label{itm:new-functions-only-defined-if-premise}
      The sequents
      \begin{equation}
        f_{\mathcal{S}, v}(v_1, \dots, v_n) \downarrow \Rightarrow \mathcal{F}
      \end{equation}
      for all variables $v$ in $\mathcal{G}$.
    \item
      \label{itm:new-functions-uniquely}
      The sequents
      \begin{equation}
        \mathcal{F} \land \mathcal{G} \implies v \equiv f_{\mathcal{S}, v}(v_1, \dots, v_n)
      \end{equation}
      for all variables $v$ in $\mathcal{G}$.
  \end{enumerate}
  Clearly algebraic sequents in $M'$ are epi.

  Now let $X$ be a relational $\mathcal{S}$-structure that satisfies all sequents in $M$.
  We define an algebraic $\mathcal{S}'$-structure $X'$ on the underlying relational $\mathcal{S}$-structure $X$ by defining
  \begin{equation}
    f_{\mathcal{S}, v}(x_1, \dots, x_n) = x
  \end{equation}
  if the assignment $v_i \mapsto x_i$ defines an interpretation of $\mathcal{F}$ in $X$, and the unique extension to an interpretation of $\mathcal{G}$ maps $v$ to $x$.
  Now clearly $X'$ satisfies all sequents in $M'$.
  
  Conversely let $X'$ be an algebraic $\mathcal{S}'$-structure satisfying all sequents in $M'$, let $X$ be the underlying relational $\mathcal{S}$-structure, and let $\mathcal{S} = \mathcal{F} \Rightarrow \mathcal{G}$ be a relational sequent in $M$.
  Then by the algebraic sequent \ref{itm:new-functions-satisfy}, $X$ satisfies $\mathcal{S}$, and the algebraic sequent \ref{itm:new-functions-uniquely}, $X$ satisfies $\mathcal{S}$ uniquely.

  Because of the sequents \ref{itm:new-functions-only-defined-if-premise}, the functions $X'_{f_{\mathcal{S}, v}}, X'_{f_{\mathcal{S}, v'}}$ have the same domain for all $v, v'$ in the premise of $\mathcal{S}$:
  The set of tuples $(x_1, \dots, x_n)$ such that $v_i \mapsto x_i$ defines an interpretation of $\mathcal{F}$ in $X$.
  Thus $X'$ is uniquely determined by $X$.

  We have thus far shown that the forgetful functor restricts to a well-defined functor $(M'^\pitchfork) \rightarrow M^\perp$ that is essentially surjective (in fact, an isomorphism on objects).
  Because every map of algebraic or relational structures is determined by its value on carriers, the forgetful functor is faithful.
  To show that it is full, we need to convince ourselves that every map $X \rightarrow Y$ of relational $\mathcal{S}$-structures satisfying all sequents in $M$ lifts to a map $X' \rightarrow Y'$, where $X'$ and $Y'$ are the algebraic structures in $(M')^\pitchfork$ mapping to $X$ and $Y$.
  This condition follows from uniqueness of lifts in diagrams of the form
  \begin{equation}
    \begin{tikzcd}
      \left[ \mathcal{F} \right] \arrow[d, "{[\mathcal{S}]}"'] \arrow[r] & X \arrow[r] & Y \\
      \left[ \mathcal{G} \right] \arrow[ur, dashed] \arrow[urr, dashed]
    \end{tikzcd}
  \end{equation}
  for relational sequents $\mathcal{S} : \mathcal{F} \Rightarrow \mathcal{G}$ in $M$.
\end{proof}

% LATEX_TO_HTML_IGNORE
\bibliographystyle{abbrvnat}
% LATEX_TO_HTML_IGNORE
\setcitestyle{authoryear,open={(},close={)}}
% LATEX_TO_HTML_IGNORE
\bibliography{main}

\end{document}
